% Generated by Sphinx.
\def\sphinxdocclass{report}
\documentclass[letterpaper,10pt,english]{sphinxmanual}
\usepackage[utf8]{inputenc}
\DeclareUnicodeCharacter{00A0}{\nobreakspace}
\usepackage[T1]{fontenc}
\usepackage{babel}
\usepackage{times}
\usepackage[Bjarne]{fncychap}
\usepackage{longtable}
\usepackage{sphinx}
\usepackage{multirow}


\title{COCOMA Documentation}
\date{April 09, 2013}
\release{1}
\author{Sergej Svorobej}
\newcommand{\sphinxlogo}{}
\renewcommand{\releasename}{Release}
\makeindex

\makeatletter
\def\PYG@reset{\let\PYG@it=\relax \let\PYG@bf=\relax%
    \let\PYG@ul=\relax \let\PYG@tc=\relax%
    \let\PYG@bc=\relax \let\PYG@ff=\relax}
\def\PYG@tok#1{\csname PYG@tok@#1\endcsname}
\def\PYG@toks#1+{\ifx\relax#1\empty\else%
    \PYG@tok{#1}\expandafter\PYG@toks\fi}
\def\PYG@do#1{\PYG@bc{\PYG@tc{\PYG@ul{%
    \PYG@it{\PYG@bf{\PYG@ff{#1}}}}}}}
\def\PYG#1#2{\PYG@reset\PYG@toks#1+\relax+\PYG@do{#2}}

\expandafter\def\csname PYG@tok@gd\endcsname{\def\PYG@tc##1{\textcolor[rgb]{0.63,0.00,0.00}{##1}}}
\expandafter\def\csname PYG@tok@gu\endcsname{\let\PYG@bf=\textbf\def\PYG@tc##1{\textcolor[rgb]{0.50,0.00,0.50}{##1}}}
\expandafter\def\csname PYG@tok@gt\endcsname{\def\PYG@tc##1{\textcolor[rgb]{0.00,0.27,0.87}{##1}}}
\expandafter\def\csname PYG@tok@gs\endcsname{\let\PYG@bf=\textbf}
\expandafter\def\csname PYG@tok@gr\endcsname{\def\PYG@tc##1{\textcolor[rgb]{1.00,0.00,0.00}{##1}}}
\expandafter\def\csname PYG@tok@cm\endcsname{\let\PYG@it=\textit\def\PYG@tc##1{\textcolor[rgb]{0.25,0.50,0.56}{##1}}}
\expandafter\def\csname PYG@tok@vg\endcsname{\def\PYG@tc##1{\textcolor[rgb]{0.73,0.38,0.84}{##1}}}
\expandafter\def\csname PYG@tok@m\endcsname{\def\PYG@tc##1{\textcolor[rgb]{0.13,0.50,0.31}{##1}}}
\expandafter\def\csname PYG@tok@mh\endcsname{\def\PYG@tc##1{\textcolor[rgb]{0.13,0.50,0.31}{##1}}}
\expandafter\def\csname PYG@tok@cs\endcsname{\def\PYG@tc##1{\textcolor[rgb]{0.25,0.50,0.56}{##1}}\def\PYG@bc##1{\setlength{\fboxsep}{0pt}\colorbox[rgb]{1.00,0.94,0.94}{\strut ##1}}}
\expandafter\def\csname PYG@tok@ge\endcsname{\let\PYG@it=\textit}
\expandafter\def\csname PYG@tok@vc\endcsname{\def\PYG@tc##1{\textcolor[rgb]{0.73,0.38,0.84}{##1}}}
\expandafter\def\csname PYG@tok@il\endcsname{\def\PYG@tc##1{\textcolor[rgb]{0.13,0.50,0.31}{##1}}}
\expandafter\def\csname PYG@tok@go\endcsname{\def\PYG@tc##1{\textcolor[rgb]{0.20,0.20,0.20}{##1}}}
\expandafter\def\csname PYG@tok@cp\endcsname{\def\PYG@tc##1{\textcolor[rgb]{0.00,0.44,0.13}{##1}}}
\expandafter\def\csname PYG@tok@gi\endcsname{\def\PYG@tc##1{\textcolor[rgb]{0.00,0.63,0.00}{##1}}}
\expandafter\def\csname PYG@tok@gh\endcsname{\let\PYG@bf=\textbf\def\PYG@tc##1{\textcolor[rgb]{0.00,0.00,0.50}{##1}}}
\expandafter\def\csname PYG@tok@ni\endcsname{\let\PYG@bf=\textbf\def\PYG@tc##1{\textcolor[rgb]{0.84,0.33,0.22}{##1}}}
\expandafter\def\csname PYG@tok@nl\endcsname{\let\PYG@bf=\textbf\def\PYG@tc##1{\textcolor[rgb]{0.00,0.13,0.44}{##1}}}
\expandafter\def\csname PYG@tok@nn\endcsname{\let\PYG@bf=\textbf\def\PYG@tc##1{\textcolor[rgb]{0.05,0.52,0.71}{##1}}}
\expandafter\def\csname PYG@tok@no\endcsname{\def\PYG@tc##1{\textcolor[rgb]{0.38,0.68,0.84}{##1}}}
\expandafter\def\csname PYG@tok@na\endcsname{\def\PYG@tc##1{\textcolor[rgb]{0.25,0.44,0.63}{##1}}}
\expandafter\def\csname PYG@tok@nb\endcsname{\def\PYG@tc##1{\textcolor[rgb]{0.00,0.44,0.13}{##1}}}
\expandafter\def\csname PYG@tok@nc\endcsname{\let\PYG@bf=\textbf\def\PYG@tc##1{\textcolor[rgb]{0.05,0.52,0.71}{##1}}}
\expandafter\def\csname PYG@tok@nd\endcsname{\let\PYG@bf=\textbf\def\PYG@tc##1{\textcolor[rgb]{0.33,0.33,0.33}{##1}}}
\expandafter\def\csname PYG@tok@ne\endcsname{\def\PYG@tc##1{\textcolor[rgb]{0.00,0.44,0.13}{##1}}}
\expandafter\def\csname PYG@tok@nf\endcsname{\def\PYG@tc##1{\textcolor[rgb]{0.02,0.16,0.49}{##1}}}
\expandafter\def\csname PYG@tok@si\endcsname{\let\PYG@it=\textit\def\PYG@tc##1{\textcolor[rgb]{0.44,0.63,0.82}{##1}}}
\expandafter\def\csname PYG@tok@s2\endcsname{\def\PYG@tc##1{\textcolor[rgb]{0.25,0.44,0.63}{##1}}}
\expandafter\def\csname PYG@tok@vi\endcsname{\def\PYG@tc##1{\textcolor[rgb]{0.73,0.38,0.84}{##1}}}
\expandafter\def\csname PYG@tok@nt\endcsname{\let\PYG@bf=\textbf\def\PYG@tc##1{\textcolor[rgb]{0.02,0.16,0.45}{##1}}}
\expandafter\def\csname PYG@tok@nv\endcsname{\def\PYG@tc##1{\textcolor[rgb]{0.73,0.38,0.84}{##1}}}
\expandafter\def\csname PYG@tok@s1\endcsname{\def\PYG@tc##1{\textcolor[rgb]{0.25,0.44,0.63}{##1}}}
\expandafter\def\csname PYG@tok@gp\endcsname{\let\PYG@bf=\textbf\def\PYG@tc##1{\textcolor[rgb]{0.78,0.36,0.04}{##1}}}
\expandafter\def\csname PYG@tok@sh\endcsname{\def\PYG@tc##1{\textcolor[rgb]{0.25,0.44,0.63}{##1}}}
\expandafter\def\csname PYG@tok@ow\endcsname{\let\PYG@bf=\textbf\def\PYG@tc##1{\textcolor[rgb]{0.00,0.44,0.13}{##1}}}
\expandafter\def\csname PYG@tok@sx\endcsname{\def\PYG@tc##1{\textcolor[rgb]{0.78,0.36,0.04}{##1}}}
\expandafter\def\csname PYG@tok@bp\endcsname{\def\PYG@tc##1{\textcolor[rgb]{0.00,0.44,0.13}{##1}}}
\expandafter\def\csname PYG@tok@c1\endcsname{\let\PYG@it=\textit\def\PYG@tc##1{\textcolor[rgb]{0.25,0.50,0.56}{##1}}}
\expandafter\def\csname PYG@tok@kc\endcsname{\let\PYG@bf=\textbf\def\PYG@tc##1{\textcolor[rgb]{0.00,0.44,0.13}{##1}}}
\expandafter\def\csname PYG@tok@c\endcsname{\let\PYG@it=\textit\def\PYG@tc##1{\textcolor[rgb]{0.25,0.50,0.56}{##1}}}
\expandafter\def\csname PYG@tok@mf\endcsname{\def\PYG@tc##1{\textcolor[rgb]{0.13,0.50,0.31}{##1}}}
\expandafter\def\csname PYG@tok@err\endcsname{\def\PYG@bc##1{\setlength{\fboxsep}{0pt}\fcolorbox[rgb]{1.00,0.00,0.00}{1,1,1}{\strut ##1}}}
\expandafter\def\csname PYG@tok@kd\endcsname{\let\PYG@bf=\textbf\def\PYG@tc##1{\textcolor[rgb]{0.00,0.44,0.13}{##1}}}
\expandafter\def\csname PYG@tok@ss\endcsname{\def\PYG@tc##1{\textcolor[rgb]{0.32,0.47,0.09}{##1}}}
\expandafter\def\csname PYG@tok@sr\endcsname{\def\PYG@tc##1{\textcolor[rgb]{0.14,0.33,0.53}{##1}}}
\expandafter\def\csname PYG@tok@mo\endcsname{\def\PYG@tc##1{\textcolor[rgb]{0.13,0.50,0.31}{##1}}}
\expandafter\def\csname PYG@tok@mi\endcsname{\def\PYG@tc##1{\textcolor[rgb]{0.13,0.50,0.31}{##1}}}
\expandafter\def\csname PYG@tok@kn\endcsname{\let\PYG@bf=\textbf\def\PYG@tc##1{\textcolor[rgb]{0.00,0.44,0.13}{##1}}}
\expandafter\def\csname PYG@tok@o\endcsname{\def\PYG@tc##1{\textcolor[rgb]{0.40,0.40,0.40}{##1}}}
\expandafter\def\csname PYG@tok@kr\endcsname{\let\PYG@bf=\textbf\def\PYG@tc##1{\textcolor[rgb]{0.00,0.44,0.13}{##1}}}
\expandafter\def\csname PYG@tok@s\endcsname{\def\PYG@tc##1{\textcolor[rgb]{0.25,0.44,0.63}{##1}}}
\expandafter\def\csname PYG@tok@kp\endcsname{\def\PYG@tc##1{\textcolor[rgb]{0.00,0.44,0.13}{##1}}}
\expandafter\def\csname PYG@tok@w\endcsname{\def\PYG@tc##1{\textcolor[rgb]{0.73,0.73,0.73}{##1}}}
\expandafter\def\csname PYG@tok@kt\endcsname{\def\PYG@tc##1{\textcolor[rgb]{0.56,0.13,0.00}{##1}}}
\expandafter\def\csname PYG@tok@sc\endcsname{\def\PYG@tc##1{\textcolor[rgb]{0.25,0.44,0.63}{##1}}}
\expandafter\def\csname PYG@tok@sb\endcsname{\def\PYG@tc##1{\textcolor[rgb]{0.25,0.44,0.63}{##1}}}
\expandafter\def\csname PYG@tok@k\endcsname{\let\PYG@bf=\textbf\def\PYG@tc##1{\textcolor[rgb]{0.00,0.44,0.13}{##1}}}
\expandafter\def\csname PYG@tok@se\endcsname{\let\PYG@bf=\textbf\def\PYG@tc##1{\textcolor[rgb]{0.25,0.44,0.63}{##1}}}
\expandafter\def\csname PYG@tok@sd\endcsname{\let\PYG@it=\textit\def\PYG@tc##1{\textcolor[rgb]{0.25,0.44,0.63}{##1}}}

\def\PYGZbs{\char`\\}
\def\PYGZus{\char`\_}
\def\PYGZob{\char`\{}
\def\PYGZcb{\char`\}}
\def\PYGZca{\char`\^}
\def\PYGZam{\char`\&}
\def\PYGZlt{\char`\<}
\def\PYGZgt{\char`\>}
\def\PYGZsh{\char`\#}
\def\PYGZpc{\char`\%}
\def\PYGZdl{\char`\$}
\def\PYGZhy{\char`\-}
\def\PYGZsq{\char`\'}
\def\PYGZdq{\char`\"}
\def\PYGZti{\char`\~}
% for compatibility with earlier versions
\def\PYGZat{@}
\def\PYGZlb{[}
\def\PYGZrb{]}
\makeatother

\begin{document}

\maketitle
\tableofcontents
\phantomsection\label{index::doc}


COCOMA framework was designed by SAP as part of EU funded BonFIRE project. Task of COCOMA framework is
to create, monitor and control contentious and malicious system workload. By using
this framework experimenters are able to make testing process more accurate
and anticipate various scenarios of cloud infrastructure behaviour, collect and
correlate metrics of the emulated environment with the test results.


\chapter{Contents}
\label{index:controlled-contentious-and-malicious-cocoma-framework-1-0}\label{index:contents}

\section{Introduction}
\label{01_how_to_use_it:introduction}\label{01_how_to_use_it::doc}
In order to use COCOMA framework experimenter creates an emulation using XML language(see below Examples section). Emulation should contain all the neccessary information
about duration, magnitude and required resource usage. Once XML document is received by COCOMA, the framework will automatically schedule and execute
required workload on the chosen resource(-s) such as CPU, IO, Memory and/or Network.


\subsection{Installation}
\label{01_how_to_use_it:installation}
The framework is designed to run on GNU/Linux and released in \emph{.deb} package only.
Once you have downloaded latest COCOMA version install it by running:

\begin{Verbatim}[commandchars=\\\{\}]
\PYGZdl{} dpkg \PYGZhy{}i cocoma\PYGZus{}X.X\PYGZhy{}X\PYGZus{}all.deb
\end{Verbatim}

The application will be installed to folder \emph{``/usr/share/pyshared/cocoma''}. All the additional required programs and libraries will be downloaded and installed on the fly if missing.
To check check if it was installed correctly run:

\begin{Verbatim}[commandchars=\\\{\}]
\PYGZdl{} ccmsh \PYGZhy{}v
\end{Verbatim}


\subsection{Starting Components}
\label{01_how_to_use_it:starting-components}
To avail full functionality of COCOMA two daemons need to be started:
\begin{itemize}
\item {} 
Scheduler daemon (mandatory)

\item {} 
API Daemon (optional if REST API functionality is required)

\end{itemize}

\textbf{Scheduler daemon} - runs in the background and executes workload with differential parameters at the time defined in the emulation properties.
to start scheduler use command:

\begin{Verbatim}[commandchars=\\\{\}]
\PYGZdl{} ccmsh \PYGZhy{}\PYGZhy{}start scheduler
\end{Verbatim}

Default network interface is \emph{eth0}, port \emph{51889} you can change that by adding required interface name and port number at the end:

\begin{Verbatim}[commandchars=\\\{\}]
\PYGZdl{} ccmsh \PYGZhy{}\PYGZhy{}start scheduler wlan0 5180
\end{Verbatim}

If more detailed output information is needed \emph{Scheduler} also can be started in \emph{DEBUG} mode:

\begin{Verbatim}[commandchars=\\\{\}]
\PYGZdl{} ccmsh \PYGZhy{}\PYGZhy{}start scheduler wlan0 5180 debug
\end{Verbatim}

\emph{Note: Scheduler needs to be running otherwise nothing will work. Always start it first!!}

\textbf{API daemon} - represents RESTfull web API which exposes COCOMA resources for use over the network. It follows the same startup pattern as the Scheduler:

\begin{Verbatim}[commandchars=\\\{\}]
\PYGZdl{} ccmsh \PYGZhy{}\PYGZhy{}start api
\end{Verbatim}

By default web API will try to start using \emph{eth0} network interface on port \emph{5050}, but it can be changed by supplying own parameters:

\begin{Verbatim}[commandchars=\\\{\}]
\PYGZdl{} ccmsh \PYGZhy{}\PYGZhy{}start api wlan0 3030
\end{Verbatim}

The log level will be always same as the \emph{Scheduler}.


\subsection{Command Line Arguments}
\label{01_how_to_use_it:command-line-arguments}
The COCOMA \textbf{ccmsh} command line interface has several options:
\index{ccmsh command line option!-h, --help}\index{-h, --help!ccmsh command line option}

\begin{fulllineitems}
\phantomsection\label{01_how_to_use_it:cmdoption-ccmsh-h}\pysigline{\bfcode{-h}\code{}\code{,~}\bfcode{--help}\code{}}
Display help information of the available options

\end{fulllineitems}

\index{ccmsh command line option!-v, --version}\index{-v, --version!ccmsh command line option}

\begin{fulllineitems}
\phantomsection\label{01_how_to_use_it:cmdoption-ccmsh-v}\pysigline{\bfcode{-v}\code{}\code{,~}\bfcode{--version}\code{}}
Display installed version information of COCOMA

\end{fulllineitems}

\index{ccmsh command line option!-l, --list \textless{}emulation name\textgreater{}}\index{-l, --list \textless{}emulation name\textgreater{}!ccmsh command line option}

\begin{fulllineitems}
\phantomsection\label{01_how_to_use_it:cmdoption-ccmsh-l}\pysigline{\bfcode{-l}\code{}\code{,~}\bfcode{--list}\code{~\textless{}emulation~name\textgreater{}}}
Display list of all emulations that are scheduled or already finished. If emulation name is provided then will it will list information for that specific emulation

\end{fulllineitems}

\index{ccmsh command line option!-r, --results \textless{}emulation name\textgreater{}}\index{-r, --results \textless{}emulation name\textgreater{}!ccmsh command line option}

\begin{fulllineitems}
\phantomsection\label{01_how_to_use_it:cmdoption-ccmsh-r}\pysigline{\bfcode{-r}\code{}\code{,~}\bfcode{--results}\code{~\textless{}emulation~name\textgreater{}}}
Display list of all emulation results that are scheduled or already finished. If emulation name is provided, then will it will list information for that specific emulation

\end{fulllineitems}

\index{ccmsh command line option!-j, --list-jobs}\index{-j, --list-jobs!ccmsh command line option}

\begin{fulllineitems}
\phantomsection\label{01_how_to_use_it:cmdoption-ccmsh-j}\pysigline{\bfcode{-j}\code{}\code{,~}\bfcode{--list-jobs}\code{}}
Querries scheduler for the list of jobs which is to be executed. Gives jobs names and planned execution time

\end{fulllineitems}

\index{ccmsh command line option!-i, --dist \textless{}distribution name\textgreater{}}\index{-i, --dist \textless{}distribution name\textgreater{}!ccmsh command line option}

\begin{fulllineitems}
\phantomsection\label{01_how_to_use_it:cmdoption-ccmsh-i}\pysigline{\bfcode{-i}\code{}\code{,~}\bfcode{--dist}\code{~\textless{}distribution~name\textgreater{}}}
Scans \emph{``/usr/share/pyshared/cocoma/distributions''} folder and displays all available distribution modules.  If distribution name is provided, then it will list help information for that specific distribution

\end{fulllineitems}

\index{ccmsh command line option!-e, --emu \textless{}emulator name\textgreater{}}\index{-e, --emu \textless{}emulator name\textgreater{}!ccmsh command line option}

\begin{fulllineitems}
\phantomsection\label{01_how_to_use_it:cmdoption-ccmsh-e}\pysigline{\bfcode{-e}\code{}\code{,~}\bfcode{--emu}\code{~\textless{}emulator~name\textgreater{}}}
Scans \emph{``/usr/share/pyshared/cocoma/emulators''} folder and displays all available emulator wrapper modules.  If emulator name is provided, then it will list help information for that specific emulator wrapper

\end{fulllineitems}

\index{ccmsh command line option!-x, --xml \textless{}file name\textgreater{}}\index{-x, --xml \textless{}file name\textgreater{}!ccmsh command line option}

\begin{fulllineitems}
\phantomsection\label{01_how_to_use_it:cmdoption-ccmsh-x}\pysigline{\bfcode{-x}\code{}\code{,~}\bfcode{--xml}\code{~\textless{}file~name\textgreater{}}}
If you have a local XML file with emulation parameters, you can use it to create emulation.

\end{fulllineitems}

\index{ccmsh command line option!-n, --now (used with -x option only)}\index{-n, --now (used with -x option only)!ccmsh command line option}

\begin{fulllineitems}
\phantomsection\label{01_how_to_use_it:cmdoption-ccmsh-n}\pysigline{\bfcode{-n}\code{}\code{,~}\bfcode{--now}\code{~(used~with~-x~option~only)}}
If your local XML file emulation has set start date in past or in future, but you want to override it and start the test right now, without modifying the file, then you can add this option after the file name i.e. \code{ccmsh -x \textless{}file name\textgreater{} -n}

\end{fulllineitems}

\index{ccmsh command line option!-d, --delete \textless{}emulation name\textgreater{}}\index{-d, --delete \textless{}emulation name\textgreater{}!ccmsh command line option}

\begin{fulllineitems}
\phantomsection\label{01_how_to_use_it:cmdoption-ccmsh-d}\pysigline{\bfcode{-d}\code{}\code{,~}\bfcode{--delete}\code{~\textless{}emulation~name\textgreater{}}}
Deletes specific emulation from the database, logs will remain and will be available until manualy deleted from \emph{``/usr/share/pyshared/cocoma/logs''} folder

\end{fulllineitems}

\index{ccmsh command line option!-p, --purge}\index{-p, --purge!ccmsh command line option}

\begin{fulllineitems}
\phantomsection\label{01_how_to_use_it:cmdoption-ccmsh-p}\pysigline{\bfcode{-p}\code{}\code{,~}\bfcode{--purge}\code{}}
Wipe all DB entries, removes all scheduled jobs, logs will remain and will be available until manualy deleted from \emph{``/usr/share/pyshared/cocoma/logs''} folder

\end{fulllineitems}

\index{ccmsh command line option!--start \textless{}api interface port\textgreater{}, \textless{}scheduler interface port\textgreater{}}\index{--start \textless{}api interface port\textgreater{}, \textless{}scheduler interface port\textgreater{}!ccmsh command line option}

\begin{fulllineitems}
\phantomsection\label{01_how_to_use_it:cmdoption-ccmsh--start}\pysigline{\bfcode{--start}\code{~\textless{}api~interface~port\textgreater{},~\textless{}scheduler~interface~port\textgreater{}}}
Launch Scheduler or API daemon by specifying network interface and port number i.e. \code{ccmsh -{-}start api eth0 2020} or \code{ccmsh -{-}start scheduler eth0 3030} . By default if interface is not specified then Scheduler daemon will run on \emph{eth0} port \emph{51889} and API daemon runs on \emph{eth0} with port \emph{5050}.

\end{fulllineitems}

\index{ccmsh command line option!--stop \textless{}api\textgreater{}, \textless{}scheduler\textgreater{}}\index{--stop \textless{}api\textgreater{}, \textless{}scheduler\textgreater{}!ccmsh command line option}

\begin{fulllineitems}
\phantomsection\label{01_how_to_use_it:cmdoption-ccmsh--stop}\pysigline{\bfcode{--stop}\code{~\textless{}api\textgreater{},~\textless{}scheduler\textgreater{}}}
Stop Scheduler or API daemon

\end{fulllineitems}

\index{ccmsh command line option!--show \textless{}api\textgreater{}, \textless{}scheduler\textgreater{}}\index{--show \textless{}api\textgreater{}, \textless{}scheduler\textgreater{}!ccmsh command line option}

\begin{fulllineitems}
\phantomsection\label{01_how_to_use_it:cmdoption-ccmsh--show}\pysigline{\bfcode{--show}\code{~\textless{}api\textgreater{},~\textless{}scheduler\textgreater{}}}
Show OS information on Scheduler or API daemon, displays PID numbers

\end{fulllineitems}



\subsection{REST API Description}
\label{01_how_to_use_it:rest-api-description}
If the web API daemon has been started successfully, then COCOMA toolkit can be accessed remotely using its RESTfull API.
\begin{itemize}
\item {} 
/

\item {} 
/emulations

\item {} 
/emulations/\{name\}

\item {} 
/distributions

\item {} 
/distributions/\{name\}

\item {} 
/emulators

\item {} 
/emulators/\{name\}

\item {} 
/results

\item {} 
/results/\{name\}

\item {} 
/tests

\item {} 
/tests/\{name\}

\item {} 
/logs

\item {} 
/logs/system

\item {} 
/logs/emulations

\item {} 
/logs/emulations/\{name\}

\end{itemize}
\index{GET (HTTP method)!/|textbf}

\begin{fulllineitems}
\phantomsection\label{01_how_to_use_it:method-get-}\pysigline{\code{GET}~\bfcode{/}}~\begin{quote}\begin{description}
\item[{Title }] \leavevmode
root

\item[{Responses}] \leavevmode\begin{itemize}
\item {} 
\textbf{200} -- OK

\item {} 
\textbf{404} -- Not Found

\end{itemize}

\end{description}\end{quote}

The \textbf{root} method returns \emph{collection} of all the available resources. Example XML response:

\begin{Verbatim}[commandchars=\\\{\}]
\PYG{c+cp}{\PYGZlt{}?xml version=\PYGZdq{}1.0\PYGZdq{} ?\PYGZgt{}}
\PYG{n+nt}{\PYGZlt{}root} \PYG{n+na}{href=}\PYG{l+s}{\PYGZdq{}/\PYGZdq{}}\PYG{n+nt}{\PYGZgt{}}
  \PYG{n+nt}{\PYGZlt{}version}\PYG{n+nt}{\PYGZgt{}}0.1.1\PYG{n+nt}{\PYGZlt{}/version\PYGZgt{}}
  \PYG{n+nt}{\PYGZlt{}timestamp}\PYG{n+nt}{\PYGZgt{}}1365518303.44\PYG{n+nt}{\PYGZlt{}/timestamp\PYGZgt{}}
  \PYG{n+nt}{\PYGZlt{}link} \PYG{n+na}{href=}\PYG{l+s}{\PYGZdq{}/emulations\PYGZdq{}} \PYG{n+na}{rel=}\PYG{l+s}{\PYGZdq{}emulations\PYGZdq{}} \PYG{n+na}{type=}\PYG{l+s}{\PYGZdq{}application/vnd.bonfire+xml\PYGZdq{}}\PYG{n+nt}{/\PYGZgt{}}
  \PYG{n+nt}{\PYGZlt{}link} \PYG{n+na}{href=}\PYG{l+s}{\PYGZdq{}/emulators\PYGZdq{}} \PYG{n+na}{rel=}\PYG{l+s}{\PYGZdq{}emulators\PYGZdq{}} \PYG{n+na}{type=}\PYG{l+s}{\PYGZdq{}application/vnd.bonfire+xml\PYGZdq{}}\PYG{n+nt}{/\PYGZgt{}}
  \PYG{n+nt}{\PYGZlt{}link} \PYG{n+na}{href=}\PYG{l+s}{\PYGZdq{}/distributions\PYGZdq{}} \PYG{n+na}{rel=}\PYG{l+s}{\PYGZdq{}distributions\PYGZdq{}} \PYG{n+na}{type=}\PYG{l+s}{\PYGZdq{}application/vnd.bonfire+xml\PYGZdq{}}\PYG{n+nt}{/\PYGZgt{}}
  \PYG{n+nt}{\PYGZlt{}link} \PYG{n+na}{href=}\PYG{l+s}{\PYGZdq{}/tests\PYGZdq{}} \PYG{n+na}{rel=}\PYG{l+s}{\PYGZdq{}tests\PYGZdq{}} \PYG{n+na}{type=}\PYG{l+s}{\PYGZdq{}application/vnd.bonfire+xml\PYGZdq{}}\PYG{n+nt}{/\PYGZgt{}}
  \PYG{n+nt}{\PYGZlt{}link} \PYG{n+na}{href=}\PYG{l+s}{\PYGZdq{}/results\PYGZdq{}} \PYG{n+na}{rel=}\PYG{l+s}{\PYGZdq{}results\PYGZdq{}} \PYG{n+na}{type=}\PYG{l+s}{\PYGZdq{}application/vnd.bonfire+xml\PYGZdq{}}\PYG{n+nt}{/\PYGZgt{}}
  \PYG{n+nt}{\PYGZlt{}link} \PYG{n+na}{href=}\PYG{l+s}{\PYGZdq{}/logs\PYGZdq{}} \PYG{n+na}{rel=}\PYG{l+s}{\PYGZdq{}logs\PYGZdq{}} \PYG{n+na}{type=}\PYG{l+s}{\PYGZdq{}application/vnd.bonfire+xml\PYGZdq{}}\PYG{n+nt}{/\PYGZgt{}}
\PYG{n+nt}{\PYGZlt{}/root\PYGZgt{}}
\end{Verbatim}

\end{fulllineitems}

\index{GET (HTTP method)!/emulations|textbf}

\begin{fulllineitems}
\phantomsection\label{01_how_to_use_it:method-get-emulations}\pysigline{\code{GET}~\bfcode{/emulations}}~\begin{quote}\begin{description}
\item[{Title }] \leavevmode
emulations

\item[{Responses}] \leavevmode\begin{itemize}
\item {} 
\textbf{200} -- OK

\item {} 
\textbf{404} -- Not Found

\end{itemize}

\end{description}\end{quote}

The \textbf{emulations} method returns \emph{collection} of all the available emulation resources. Example XML response:

\begin{Verbatim}[commandchars=\\\{\}]
\PYG{c+cp}{\PYGZlt{}?xml version=\PYGZdq{}1.0\PYGZdq{} ?\PYGZgt{}}
 \PYG{n+nt}{\PYGZlt{}collection} \PYG{n+na}{href=}\PYG{l+s}{\PYGZdq{}/emulations\PYGZdq{}} \PYG{n+na}{xmlns=}\PYG{l+s}{\PYGZdq{}http://127.0.0.1/cocoma\PYGZdq{}}\PYG{n+nt}{\PYGZgt{}}
   \PYG{n+nt}{\PYGZlt{}items} \PYG{n+na}{offset=}\PYG{l+s}{\PYGZdq{}0\PYGZdq{}} \PYG{n+na}{total=}\PYG{l+s}{\PYGZdq{}3\PYGZdq{}}\PYG{n+nt}{\PYGZgt{}}
     \PYG{n+nt}{\PYGZlt{}emulation} \PYG{n+na}{href=}\PYG{l+s}{\PYGZdq{}/emulations/1\PYGZhy{}Emu\PYGZhy{}CPU\PYGZhy{}RAM\PYGZhy{}IO\PYGZdq{}} \PYG{n+na}{id=}\PYG{l+s}{\PYGZdq{}1\PYGZdq{}} \PYG{n+na}{name=}\PYG{l+s}{\PYGZdq{}1\PYGZhy{}Emu\PYGZhy{}CPU\PYGZhy{}RAM\PYGZhy{}IO\PYGZdq{}} \PYG{n+na}{state=}\PYG{l+s}{\PYGZdq{}inactive\PYGZdq{}}\PYG{n+nt}{/\PYGZgt{}}
     \PYG{n+nt}{\PYGZlt{}emulation} \PYG{n+na}{href=}\PYG{l+s}{\PYGZdq{}/emulations/2\PYGZhy{}CPU\PYGZus{}EMU\PYGZdq{}} \PYG{n+na}{id=}\PYG{l+s}{\PYGZdq{}2\PYGZdq{}} \PYG{n+na}{name=}\PYG{l+s}{\PYGZdq{}2\PYGZhy{}CPU\PYGZus{}EMU\PYGZdq{}} \PYG{n+na}{state=}\PYG{l+s}{\PYGZdq{}inactive\PYGZdq{}}\PYG{n+nt}{/\PYGZgt{}}
     \PYG{n+nt}{\PYGZlt{}emulation} \PYG{n+na}{href=}\PYG{l+s}{\PYGZdq{}/emulations/3\PYGZhy{}CPU\PYGZus{}EMU\PYGZdq{}} \PYG{n+na}{id=}\PYG{l+s}{\PYGZdq{}3\PYGZdq{}} \PYG{n+na}{name=}\PYG{l+s}{\PYGZdq{}3\PYGZhy{}CPU\PYGZus{}EMU\PYGZdq{}} \PYG{n+na}{state=}\PYG{l+s}{\PYGZdq{}inactive\PYGZdq{}}\PYG{n+nt}{/\PYGZgt{}}
   \PYG{n+nt}{\PYGZlt{}/items\PYGZgt{}}
   \PYG{n+nt}{\PYGZlt{}link} \PYG{n+na}{href=}\PYG{l+s}{\PYGZdq{}/\PYGZdq{}} \PYG{n+na}{rel=}\PYG{l+s}{\PYGZdq{}parent\PYGZdq{}} \PYG{n+na}{type=}\PYG{l+s}{\PYGZdq{}application/vnd.bonfire+xml\PYGZdq{}}\PYG{n+nt}{/\PYGZgt{}}
 \PYG{n+nt}{\PYGZlt{}/collection\PYGZgt{}}
\end{Verbatim}

\end{fulllineitems}

\index{GET (HTTP method)!/emulations/\{name\}|textbf}

\begin{fulllineitems}
\phantomsection\label{01_how_to_use_it:method-get-emulations-name-}\pysigline{\code{GET}~\bfcode{/emulations/\emph{\{name\}}}}~\begin{quote}\begin{description}
\item[{Path arguments}] \leavevmode
\textbf{name} -- Name of emulation that you want to get more info

\item[{Responses}] \leavevmode\begin{itemize}
\item {} 
\textbf{200} -- OK

\item {} 
\textbf{404} -- Not Found

\end{itemize}

\end{description}\end{quote}

Displays information about emulation by name. The returned \emph{200-OK} XML is:

\begin{Verbatim}[commandchars=\\\{\}]
\PYG{c+cp}{\PYGZlt{}?xml version=\PYGZdq{}1.0\PYGZdq{} ?\PYGZgt{}}
\PYG{n+nt}{\PYGZlt{}emulation} \PYG{n+na}{href=}\PYG{l+s}{\PYGZdq{}/emulations/1\PYGZhy{}Emu\PYGZhy{}CPU\PYGZhy{}RAM\PYGZhy{}IO\PYGZdq{}} \PYG{n+na}{xmlns=}\PYG{l+s}{\PYGZdq{}http://127.0.0.1/cocoma\PYGZdq{}}\PYG{n+nt}{\PYGZgt{}}
  \PYG{n+nt}{\PYGZlt{}id}\PYG{n+nt}{\PYGZgt{}}1\PYG{n+nt}{\PYGZlt{}/id\PYGZgt{}}
  \PYG{n+nt}{\PYGZlt{}emulationName}\PYG{n+nt}{\PYGZgt{}}1\PYGZhy{}Emu\PYGZhy{}CPU\PYGZhy{}RAM\PYGZhy{}IO\PYG{n+nt}{\PYGZlt{}/emulationName\PYGZgt{}}
  \PYG{n+nt}{\PYGZlt{}emulationType}\PYG{n+nt}{\PYGZgt{}}mix\PYG{n+nt}{\PYGZlt{}/emulationType\PYGZgt{}}
  \PYG{n+nt}{\PYGZlt{}resourceType}\PYG{n+nt}{\PYGZgt{}}mix\PYG{n+nt}{\PYGZlt{}/resourceType\PYGZgt{}}
  \PYG{n+nt}{\PYGZlt{}emuStartTime}\PYG{n+nt}{\PYGZgt{}}2013\PYGZhy{}04\PYGZhy{}09T13:00:01\PYG{n+nt}{\PYGZlt{}/emuStartTime\PYGZgt{}}
  \PYG{n+nt}{\PYGZlt{}emuStopTime}\PYG{n+nt}{\PYGZgt{}}180\PYG{n+nt}{\PYGZlt{}/emuStopTime\PYGZgt{}}
  \PYG{n+nt}{\PYGZlt{}scheduledJobs}\PYG{n+nt}{\PYGZgt{}}
    \PYG{n+nt}{\PYGZlt{}jobsempty}\PYG{n+nt}{\PYGZgt{}}No jobs are scheduled\PYG{n+nt}{\PYGZlt{}/jobsempty\PYGZgt{}}
  \PYG{n+nt}{\PYGZlt{}/scheduledJobs\PYGZgt{}}
  \PYG{n+nt}{\PYGZlt{}distributions} \PYG{n+na}{ID=}\PYG{l+s}{\PYGZdq{}1\PYGZdq{}} \PYG{n+na}{name=}\PYG{l+s}{\PYGZdq{}Distro1\PYGZdq{}}\PYG{n+nt}{\PYGZgt{}}
    \PYG{n+nt}{\PYGZlt{}startTime}\PYG{n+nt}{\PYGZgt{}}5\PYG{n+nt}{\PYGZlt{}/startTime\PYGZgt{}}
    \PYG{n+nt}{\PYGZlt{}granularity}\PYG{n+nt}{\PYGZgt{}}3\PYG{n+nt}{\PYGZlt{}/granularity\PYGZgt{}}
    \PYG{n+nt}{\PYGZlt{}duration}\PYG{n+nt}{\PYGZgt{}}30\PYG{n+nt}{\PYGZlt{}/duration\PYGZgt{}}
    \PYG{n+nt}{\PYGZlt{}startload}\PYG{n+nt}{\PYGZgt{}}10\PYG{n+nt}{\PYGZlt{}/startload\PYGZgt{}}
    \PYG{n+nt}{\PYGZlt{}stopload}\PYG{n+nt}{\PYGZgt{}}90\PYG{n+nt}{\PYGZlt{}/stopload\PYGZgt{}}
  \PYG{n+nt}{\PYGZlt{}/distributions\PYGZgt{}}
  \PYG{n+nt}{\PYGZlt{}distributions} \PYG{n+na}{ID=}\PYG{l+s}{\PYGZdq{}2\PYGZdq{}} \PYG{n+na}{name=}\PYG{l+s}{\PYGZdq{}Distro2\PYGZdq{}}\PYG{n+nt}{\PYGZgt{}}
    \PYG{n+nt}{\PYGZlt{}startTime}\PYG{n+nt}{\PYGZgt{}}5\PYG{n+nt}{\PYGZlt{}/startTime\PYGZgt{}}
    \PYG{n+nt}{\PYGZlt{}granularity}\PYG{n+nt}{\PYGZgt{}}3\PYG{n+nt}{\PYGZlt{}/granularity\PYGZgt{}}
    \PYG{n+nt}{\PYGZlt{}duration}\PYG{n+nt}{\PYGZgt{}}30\PYG{n+nt}{\PYGZlt{}/duration\PYGZgt{}}
    \PYG{n+nt}{\PYGZlt{}startload}\PYG{n+nt}{\PYGZgt{}}10\PYG{n+nt}{\PYGZlt{}/startload\PYGZgt{}}
    \PYG{n+nt}{\PYGZlt{}stopload}\PYG{n+nt}{\PYGZgt{}}90\PYG{n+nt}{\PYGZlt{}/stopload\PYGZgt{}}
  \PYG{n+nt}{\PYGZlt{}/distributions\PYGZgt{}}
  \PYG{n+nt}{\PYGZlt{}link} \PYG{n+na}{href=}\PYG{l+s}{\PYGZdq{}/\PYGZdq{}} \PYG{n+na}{rel=}\PYG{l+s}{\PYGZdq{}parent\PYGZdq{}} \PYG{n+na}{type=}\PYG{l+s}{\PYGZdq{}application/vnd.bonfire+xml\PYGZdq{}}\PYG{n+nt}{/\PYGZgt{}}
  \PYG{n+nt}{\PYGZlt{}link} \PYG{n+na}{href=}\PYG{l+s}{\PYGZdq{}/emulations\PYGZdq{}} \PYG{n+na}{rel=}\PYG{l+s}{\PYGZdq{}parent\PYGZdq{}} \PYG{n+na}{type=}\PYG{l+s}{\PYGZdq{}application/vnd.bonfire+xml\PYGZdq{}}\PYG{n+nt}{/\PYGZgt{}}
\PYG{n+nt}{\PYGZlt{}/emulation\PYGZgt{}}
\end{Verbatim}

The returned \emph{404 – Not Found} XML is:

\begin{Verbatim}[commandchars=\\\{\}]
\PYG{n+nt}{\PYGZlt{}error}\PYG{n+nt}{\PYGZgt{}}Emulation Name: 1\PYGZhy{}Emu\PYGZhy{}CPU\PYGZhy{}RAM\PYGZhy{}IO1 not found. Error:too many values to unpack\PYG{n+nt}{\PYGZlt{}/error\PYGZgt{}}
\end{Verbatim}

\end{fulllineitems}

\index{POST (HTTP method)!/emulations|textbf}

\begin{fulllineitems}
\phantomsection\label{01_how_to_use_it:method-post-emulations}\pysigline{\code{POST}~\bfcode{/emulations}}~\begin{quote}\begin{description}
\item[{Query params}] \leavevmode
\textbf{XML} (\emph{string}) -- Emulation parameters defined via XML as shown in the examples section.

\item[{Responses}] \leavevmode\begin{itemize}
\item {} 
\textbf{201} -- Emulation was created successfully

\item {} 
\textbf{400} -- Bad Request

\end{itemize}

\end{description}\end{quote}

Create emulation.

\end{fulllineitems}

\index{GET (HTTP method)!/emulators|textbf}

\begin{fulllineitems}
\phantomsection\label{01_how_to_use_it:method-get-emulators}\pysigline{\code{GET}~\bfcode{/emulators}}~\begin{quote}\begin{description}
\item[{Responses}] \leavevmode\begin{itemize}
\item {} 
\textbf{200} -- OK

\item {} 
\textbf{404} -- Not Found

\end{itemize}

\end{description}\end{quote}

Displays emulators list.

\end{fulllineitems}

\index{GET (HTTP method)!/emulators/\{name\}|textbf}

\begin{fulllineitems}
\phantomsection\label{01_how_to_use_it:method-get-emulators-name-}\pysigline{\code{GET}~\bfcode{/emulators/\emph{\{name\}}}}~\begin{quote}\begin{description}
\item[{Path arguments}] \leavevmode
\textbf{name} -- Name of emulator that you want to get more info

\item[{Responses}] \leavevmode\begin{itemize}
\item {} 
\textbf{200} -- OK

\item {} 
\textbf{404} -- Not Found

\end{itemize}

\end{description}\end{quote}

Displays information about emulator by name.

\end{fulllineitems}

\index{GET (HTTP method)!/distributions|textbf}

\begin{fulllineitems}
\phantomsection\label{01_how_to_use_it:method-get-distributions}\pysigline{\code{GET}~\bfcode{/distributions}}~\begin{quote}\begin{description}
\item[{Responses}] \leavevmode\begin{itemize}
\item {} 
\textbf{200} -- OK

\item {} 
\textbf{404} -- Not Found

\end{itemize}

\end{description}\end{quote}

Displays distributions list.

\end{fulllineitems}

\index{GET (HTTP method)!/distributions/\{name\}|textbf}

\begin{fulllineitems}
\phantomsection\label{01_how_to_use_it:method-get-distributions-name-}\pysigline{\code{GET}~\bfcode{/distributions/\emph{\{name\}}}}~\begin{quote}\begin{description}
\item[{Path arguments}] \leavevmode
\textbf{name} -- Name of distributions that you want to get more info

\item[{Responses}] \leavevmode\begin{itemize}
\item {} 
\textbf{200} -- OK

\item {} 
\textbf{404} -- Not Found

\end{itemize}

\end{description}\end{quote}

Displays information about distributions by name.

\end{fulllineitems}

\index{GET (HTTP method)!/tests|textbf}

\begin{fulllineitems}
\phantomsection\label{01_how_to_use_it:method-get-tests}\pysigline{\code{GET}~\bfcode{/tests}}~\begin{quote}\begin{description}
\item[{Responses}] \leavevmode\begin{itemize}
\item {} 
\textbf{200} -- OK

\item {} 
\textbf{404} -- Not Found

\end{itemize}

\end{description}\end{quote}

Displays tests list.

\end{fulllineitems}

\index{GET (HTTP method)!/tests/\{name\}|textbf}

\begin{fulllineitems}
\phantomsection\label{01_how_to_use_it:method-get-tests-name-}\pysigline{\code{GET}~\bfcode{/tests/\emph{\{name\}}}}~\begin{quote}\begin{description}
\item[{Path arguments}] \leavevmode
\textbf{name} -- Name of tests that you want to get more info

\item[{Responses}] \leavevmode\begin{itemize}
\item {} 
\textbf{200} -- OK

\item {} 
\textbf{404} -- Not Found

\end{itemize}

\end{description}\end{quote}

Displays information about tests by name.

\end{fulllineitems}

\index{POST (HTTP method)!/tests/\{name\}|textbf}

\begin{fulllineitems}
\phantomsection\label{01_how_to_use_it:method-post-tests-name-}\pysigline{\code{POST}~\bfcode{/tests/\emph{\{name\}}}}~\begin{quote}\begin{description}
\item[{Query params}] \leavevmode
\textbf{string} -- name of the test that is located on COCOMA server

\item[{Responses}] \leavevmode\begin{itemize}
\item {} 
\textbf{201} -- Emulation was created successfully

\item {} 
\textbf{400} -- Bad Request

\end{itemize}

\end{description}\end{quote}

Create emulation from already available tests

\end{fulllineitems}

\index{GET (HTTP method)!/results|textbf}

\begin{fulllineitems}
\phantomsection\label{01_how_to_use_it:method-get-results}\pysigline{\code{GET}~\bfcode{/results}}~\begin{quote}\begin{description}
\item[{Responses}] \leavevmode\begin{itemize}
\item {} 
\textbf{200} -- OK

\item {} 
\textbf{404} -- Not Found

\end{itemize}

\end{description}\end{quote}

Displays results list.

\end{fulllineitems}

\index{GET (HTTP method)!/results/\{name\}|textbf}

\begin{fulllineitems}
\phantomsection\label{01_how_to_use_it:method-get-results-name-}\pysigline{\code{GET}~\bfcode{/results/\emph{\{name\}}}}~\begin{quote}\begin{description}
\item[{Path arguments}] \leavevmode
\textbf{name} -- Name of tests that you want to get more info

\item[{Responses}] \leavevmode\begin{itemize}
\item {} 
\textbf{200} -- OK

\item {} 
\textbf{404} -- Not Found

\end{itemize}

\end{description}\end{quote}

Displays information about results by name.

\end{fulllineitems}

\index{GET (HTTP method)!/logs/system|textbf}

\begin{fulllineitems}
\phantomsection\label{01_how_to_use_it:method-get-logs-system}\pysigline{\code{GET}~\bfcode{/logs/system}}~\begin{quote}\begin{description}
\item[{Responses}] \leavevmode\begin{itemize}
\item {} 
\textbf{200} -- OK

\item {} 
\textbf{404} -- Not Found

\end{itemize}

\end{description}\end{quote}

Return Zip file with system logs.

\end{fulllineitems}

\index{GET (HTTP method)!/logs/emulations|textbf}

\begin{fulllineitems}
\phantomsection\label{01_how_to_use_it:method-get-logs-emulations}\pysigline{\code{GET}~\bfcode{/logs/emulations}}~\begin{quote}\begin{description}
\item[{Responses}] \leavevmode\begin{itemize}
\item {} 
\textbf{200} -- OK

\item {} 
\textbf{404} -- Not Found

\end{itemize}

\end{description}\end{quote}

Displays emulations logs list.

\end{fulllineitems}

\index{GET (HTTP method)!/logs/\{name\}|textbf}

\begin{fulllineitems}
\phantomsection\label{01_how_to_use_it:method-get-logs-name-}\pysigline{\code{GET}~\bfcode{/logs/\emph{\{name\}}}}~\begin{quote}\begin{description}
\item[{Path arguments}] \leavevmode
\textbf{name} -- Name of emulation logs that you want to get more info

\item[{Responses}] \leavevmode\begin{itemize}
\item {} 
\textbf{200} -- OK

\item {} 
\textbf{404} -- Not Found

\end{itemize}

\end{description}\end{quote}

Return Zip file with emulation logs.

\end{fulllineitems}



\chapter{Indices and tables}
\label{index:indices-and-tables}\begin{itemize}
\item {} 
\emph{genindex}

\item {} 
\emph{search}

\end{itemize}



\renewcommand{\indexname}{Index}
\printindex
\end{document}
